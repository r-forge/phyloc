\documentclass{article}
%\VignetteIndexEntry{phylo4 design}
%\VignettePackage{phylo4}
%\VignetteDepends{ape}
\usepackage[utf8]{inputenc} % for UTF-8/single quotes from sQuote()
\newcommand{\code}[1]{{{\tt #1}}}
\title{Design decisions for the phylo4 class}
\author{Ben Bolker}
\date{\today}
\usepackage{/usr/share/R/share/texmf/Sweave}
\begin{document}
\maketitle

This document describes the design decisions associated
with the new ``phylo4'' class, which is intended to
provide some kind of unifying standard for phylogenetic
data in R.  It is closely modeled on the the \code{phylo}
class in \code{ape}, which is the dominant data structure
at the moment.

Like \code{phylo}, the main components of
the \code{phylo4} class are:
\begin{description}
\item[edge]{an $N \times 2$ matrix of integers,
  where the first column \ldots}
\item[Nnode]{integer, number of nodes}
\item[tip.label]{character vector of tip labels}
\item[root.edge]{integer or NA, defining root edge}
\end{description}
 
\section{Hacks/backward compatibility}

Hilmar Lapp very kindly showed a way to hack
the \verb+$+ operator so that it would provide
backward compatibility with code that is 
extracting internal elements of a \code{phylo4}.

\begin{Schunk}
\begin{Sinput}
> library(ape)
\end{Sinput}
\end{Schunk}
\section{To do/problems}


\begin{itemize}
\item Warning on undocumented code objects: print, nNodes, nTips, \ldots
\end{itemize}

\end{document}
