\documentclass{article}
%\VignetteIndexEntry{phylo4 design}
%\VignettePackage{phylo4}
%\VignetteDepends{ape}
\usepackage[utf8]{inputenc} % for UTF-8/single quotes from sQuote()
\newcommand{\code}[1]{{{\tt #1}}}
\title{Design decisions for the phylo4 class}
\author{Ben Bolker}
\date{\today}
\usepackage{/usr/share/R/share/texmf/Sweave}
\begin{document}
\maketitle

This document describes the design decisions associated
with the new ``phylo4'' class, which is intended to
provide some kind of unifying standard for phylogenetic
data in R.  It is closely modeled on the the \code{phylo}
class in \code{ape}, which is the dominant data structure
at the moment.

\section{Definitions/slots}

\subsection{phylo4}
Like \code{phylo}, the main components of
the \code{phylo4} class are:
\begin{description}
\item[edge]{an $N \times 2$ matrix of integers,
  where the first column \ldots}
\item[edge.length]{numeric list of edge lengths
(length $N$ or empty)}
\item[Nnode]{integer, number of nodes}
\item[tip.label]{character vector of tip labels (required)}
\item[node.label]{character vector of node labels (maybe empty)}
\item[root.edge]{integer defining root edge (maybe NA)}
\end{description}

We have defined basic methods for \code{phylo4}:
\code{show}, \code{print} (copied from \code{print.phylo} in
\code{ape}), and a variety of accessor functions (see help files).

\code{summary} does not seem to be terribly useful in the
context of a ``raw'' tree, because there is not much to
compute: \textbf{end users?}

Print method: add information about
(ultrametric, scaled, polytomies (zero-length or structural))?

\subsection{phylo4d}

The \code{phylo4d} class extends \code{phylo4}
with data.  Tip data, (internal) node data, and edge
data are stored separately, but can be retrieved
together or separately with \code{tdata(x,"tip")}
or \code{tdata(x,"all")}.

\textbf{edge data can also be included --- is this
useful/worth keeping?}
 
\section{Hacks/backward compatibility}

Hilmar Lapp very kindly showed a way to hack
the \verb+$+ operator so that it would provide
backward compatibility with code that is 
extracting internal elements of a \code{phylo4}.
The basic recipe is:
\begin{Schunk}
\begin{Sinput}
> setMethod("$", "phylo4", function(x, name) {
+     attr(x, name)
+ })
\end{Sinput}
\end{Schunk}
but this has to be hacked slightly to intercept
calls to elements that might be missing.  For example,
\code{ape} detects whether log-likelihood, root edges,
node labels, etc. are missing by testing whether they
are \code{NULL}, whereas missing items are represented
in \code{phylo4} by zero-length vectors in the slots 
(or \code{NA} for the root edge) --- so we need code
like
\begin{Schunk}
\begin{Sinput}
> if (!hasNodeLabels(x)) NULL else x@node.label
\end{Sinput}
\end{Schunk}
to handle these cases.

\begin{Schunk}
\begin{Sinput}
> library(ape)
\end{Sinput}
\end{Schunk}
\section{To do/problems}

\begin{itemize}
\item Conflict with \code{nTips} if \code{ape} is loaded first:
ask EP to get rid of this (obsolete?) function? (\code{Ntips}
is the real \code{ape} function for getting the number of tips)
\item basic tree manipulation: tip-dropping, \code{na.omit}, etc. --- 
  especially for multi-tree and tree-with-data cases
\item tree-manipulation code: tree traversal (store current
  position as an attribute), pruning, etc.
\item restrict/specify edges matrix to be integer?
\end{itemize}

\end{document}
